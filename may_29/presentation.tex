\documentclass[xcolor=dvipsnames,12pt]{beamer}
\usepackage{tikz}
\usepackage{url}
\usepackage{biblatex}
\usepackage{tabularx}
\usepackage{multirow}
\usepackage{listings}
\usecolortheme[named=Plum]{structure} 
\usetheme[height=10mm]{Rochester} 
\setbeamertemplate{navigation symbols}{} 
%\title[Discussion ]{Introduction  to Pacemaker}
\author{Paulius and Bogdan}
\date{May 29, 2013}
\bibliography{refs}
\newcommand\Fontvi{\fontsize{14}{14}\selectfont}

\begin{document}
  \begin{frame}
    \frametitle{Meeting agenda}
    \Fontvi
    \begin{enumerate}
      \item New developments in translation to Alloy
      \item Design choices in the semantics of behavioral Clafer
        \begin{itemize}
          \item Meaning of current structural(cross-tree) constraints
          \item Default behavior of subclafers: mutable vs immutable
        \end{itemize}
      \item{Design choices in the concrete syntax}
    \end{enumerate}
  \end{frame}

  \begin{frame}
    \frametitle{Translation to Alloy}
    We reconsidered translation to Alloy. Issues using Amir's vanilla library:
    \begin{itemize}
      \item{Need for global state}
      \item{Issues with identity when cardinality is more than 1}
      \item{Not compatible with current compiler Alloy output}
      \item{Logical expression require use of library functions, instead of using Alloy operators}
      \item{Suffers from state explosion}
    \end{itemize}

  \end{frame}

  \begin{frame}
    \frametitle{New solution}
    New solution is still similar and based on Bounded Model Checking with Alloy paper \footfullcite{DBLP:journals/corr/abs-1207-2746}. Instead of global state we introduce local state concept:
    \begin{enumerate}
      \item{Define discrete Time ordered using \lstinline{util/ordering} module.}
      \item{Since Time set is finite we add a loop relation from last Time instance to any other one.}
      \item{Each mutable field relation gets additional Time column.}
      \item{Define behavioral constraints using LTL. LTL encoding over Time is presented in the paper.}
      \item{Traces are modeled according to the ordering of Time atoms. A snapshot in a trace is assembly of immutable values and projection of mutable values at specific Time instance.}
    \end{enumerate}
  \end{frame}


  \begin{frame}[fragile]
    \frametitle{Meaning of current cross-tree constraints}
    Current cross-tree constraints may have two different semantics in behavioral Clafer.

    \mbox{}

    \begin{columns}[t]
      \column{2in}
      Restricts the first state
      \begin{itemize}
        \item Similar to LTL/CTL
        \item Often meant to restrict all states, so models will need to be altered with global modalities
      \end{itemize}
    \column{2in}
    Restrict globally
    \begin{itemize}
      \item{Easy to restrict all states in the trace}
      \item{Different semantics from LTL/CTL, so temporal constraints need new concrete syntax}
      \item{Otherwise hard to reason about initial states}
      \end{itemize}
    \end{columns}    
\end{frame}

  \begin{frame}[fragile]
    \frametitle{Subclafer mutability}
    It can be difficult to implicitly imply which subclafers are mutable. Therefore we need some kind of assumption about default mutability and concrete syntax to express opposite.
    \begin{itemize}
      \item Should top level clafers be immutable? 
      \item Should we imply that subclafers are immutable or mutable by default?
    \end{itemize}
    \begin{columns}[t]
      \column{1.8in}
      All fields are immutable by default
\begin{verbatim}
PM
  heart -> Heart
CaseHandler
[mutable] current -> Case
\end{verbatim}
\column{1.8in}
All fields are mutable by default
\begin{verbatim}
Person
  name: String
  [immutable name]
Person
  age: int
\end{verbatim}
\end{columns}    
\end{frame}

\end{document}




